\documentclass[11pt]{amsart}
\usepackage[utf8]{inputenc}
\usepackage{geometry}                % See geometry.pdf to learn the layout options. There are lots.
\geometry{letterpaper}                   % ... or a4paper or a5paper or ... 
%\geometry{landscape}                % Activate for for rotated page geometry
%\usepackage[parfill]{parskip}    % Activate to begin paragraphs with an empty line rather than an indent
\usepackage{graphicx}
\usepackage{amssymb}
\usepackage{epstopdf}
\DeclareGraphicsRule{.tif}{png}{.png}{`convert #1 `dirname #1`/`basename #1 .tif`.png}

\title{Humanoider Roboter - Spezifikation}
\author{Jenning Sch\"afer}
%\date{}                                           % Activate to display a given date or no date

\begin{document}
\maketitle
\tableofcontents
\section{Allgemeine Beschreibung}
In diesem Projekt soll ein humanoider Roboter gebaut und Software erstellt werden, welche sich beide von anderen Projekten durch einen hohen Grad an Modularit\"at, Flexibilit\"at und Skalierbarkeit zu unterscheiden. Dabei sollen alle Baupl\"ane ver\"offentlicht und frei zug\"anglich sein und die Software nach GPL lizensiert werden, ob GPLv2, GPLv2 or later oder GPLv3 soll noch diskutiert werden um m\"oglichst offen und zug\"anglich zu sein. 

Anlass ist die Idee, dass ein derart komplexes Projekt, wie die Erstellung eines Humanoiden Roboter mit brauchbarer Intelligenz nicht von 20 Wissenschaftlern in einem Labor und schon gar nicht, wenn jedes Projekt von vorne anf\"fangt realisiert werden kann. Stattdessen ben\"otigt es ein Projekt, das weltweit Standard wird, so dass dies immer als Referenz oder als Grundlage genommen werden kann, was dank Copyleft das Projekt wieder voran treibt.

\section{Funktionsweise}
\begin{description}
   \item[Das alleinstellungsmerkmal sind folgende Funktionen]~\par
   \begin{enumerate}
      \item Software
      \begin{enumerate}
         \item Aufbau als verteiltes System
         \item Hoher Grad an Skalierbarkeit
         \item Der Roboter soll Autonom sein
         \item Der Roboter soll sich mit anderen Roboter in einem Peer-to-Peer Netzwerk befinden k\"onnen um wissen zu teilen
         \item das Netzwerk dazu benutzen k\"onnen um Rechenleistung  zu teilen wie ein Cluster Computer
         \item Einen Server als Angelpunkt verwenden k\"innen auf dem ebenfalls Rechenarbeit abgegeben werden kann. 
         \item Die Serversoftware soll wiederum auch als Clustercomputer benutzt werden k\"onnen.
         \item Jede Zeile Code die ver\"ofentlicht wird soll sauber geschrieben und gut dokumentiert sein. Denn ein Hack zieht viele nach sich und macht den Code auf Dauer unbenutzbar
         \item Gr\"o\ss{}tm\"ogliche Sicherheit
         \item einfach zu updaten
      \end{enumerate}
      \item Hardware
         \begin{enumerate}
         		\item Einzelne Bauteile wie Arm Kopf und Beine sollen mit und ohne dem Rest Funktionieren k\"onnen
         		\item Teile sollen Austauschbar sein
      	\end{enumerate}
      \item Meilensteine aus technischer Sicht
      	\begin{enumerate}
		\item Kick-Off Meeting mit allen beteiligten
		\item Verfeinerte Spezifikation
		\item Vertraut machen mit der Technologie
		\item Roboter entwerfen
		\item Roboter bauen und ihn zum laufen bringen
		\item Roboter \"uber den Computer Steuern (mit der Tastatur in verschiedene Richtungen bewegen)
		\item An die Community \"ubergeben
      	\end{enumerate}
   \end{enumerate}
\end{description}

\subsection{Software}
\subsubsection{Aufbau als verteiltes System}
Das Roboter-System soll nicht nur bez\"uglich Server und der Verbindung zu anderen Ger\"aten als verteiltes System aufgebaut sein, sondern auch in Bezug auch sich selbst. Damit sollte es kein Problem darstellen teile auszutauschen, einzeln zu benutzen und eine Infrastruktur f\"ur noch unbekannte zuk\"unftige Anwendungsm\"oglichkeiten bereit zu stellen.
\subsubsection{Hoher Grad an Skalierbarkeit}
Das Roboter-System sollte eine Infrastruktur bereitstellen, die es dem Anwender erlaubt das System mit so viel Ger\"aten wie m\"oglich zu verbinden. Ob nun ein Roboter mit einem Server, anderen Modulen, Module untereinander, Roboter miteinander, eine Hausautomation oder das Roboter-System mit Hauautomation. Prinzipiell sollte die Architektur von Anfang an so aufgebaut sein, dass diesbez\"uglich so wenig Barrieren wie m\"oglich und entsprechende Schnittstellen geschaffen werden oder hinzugef\"ugt werden k\"onnen. Anders formuliert, f\"ur das System sollte es egal sein wie viele oder welche Ger\"ate miteinander kommunizieren. Bez\"uglich der Regelung und Steuerung dieser Ger\"ate wird noch diskutiert.
\subsubsection{Autonimit\"at}
Obwohl das Roboter-System prinzipiell als verteiltes System entworfen wird, so sollte ein Einzelner Roboter trotzdem so konzipiert werden, dass er im Bedarfsfall Autonom aufgebaut wird und bestimmte Handlungen autonom, hei\ss{}t unabh\"angig vom Netzwerk ausf\"uhrt. Soll der Roboter beispielsweise komplett autonom sein, so sollte es m\"oglich sein ihn mit h\"oherer Rechenleistung auszustatten. Prinzipiell Handlungen die autonom ausgef\"uhrt werden sollten, sollten Handlungen wie das verhindern von fallen bei Gleichgewichtsverlust sein, um die Reaktionszeit zu verk\"urzen. 
\subsubsection{Peer-to-Peer}
Trotzdem der m\"oglichkeit einen Server in das System einzubauen um von dort den Roboter zu steuern, dort Informationen zu speichern um diese im Netzwerk verf\"ugbar zu machen, sollte die Architektur nicht auf Client-Server sondern auf eine Peer-to-Peer Architektur setzen. So k\"onnen Module (alle Ger\"ate, Roboter, etc. im System werden als Module betrachtet) direkt ohne Umweg \"uber den Server miteinander kommunizieren und, falls es nur einen WLAN Hotspot gibt, auch als AccesPoint f\"ur das Netzwerk benutzt werden, da dieses Modul auch ins Netzwerk integriert ist.
\\
\\
*Randnotiz an Armin und Bastl:
Frage: Ein Server kann ja auch benutzt werden um einen Roboter zu steuern, was nun wenn jemand auf die Idee kommt und sagt, P2P, geil, schlie\ss{} ich  mehrere Server an? Und was wenn er ein Scherzkeks ist und einen Roboter mit zwei Servern gleichzeitig zwei verschiedene Dinge tun lassen will? Oder, und ich glaub schon, dass es Menschen gibt die so was ausprobieren, einfach weil ses k\"onnen, wenn zwei Roboter Netzwerk sich \"uber das Internet verbinden um Erfahrungen auszutauschen, vielleicht auch hier nur bestimmte Daten.  Ich mein so n Spa\ss{} wie die zus\"atzliche Steuerung von Mobilen Ger\"alten interessiert mich in der Frage wenig, weil des seh ich mehr als Remote einem bestimmten Server an. Jetzt die eigentliche Frage, w\"are es sinnvoll verschiedenen Modulen eine Gruppe zuzuweisen, f\"ur die es sozusagen einen Master\&Commander Server gibt, der im Zweifelsfall immer die Befehlsgewalt hat? Und w\"are es dann auch m\"oglich Server in Gruppen zu unterteilen \"uber die ein anderer Server die Befehlsgewalt hat? Wei\ss{} nicht wer des warum braucht, aber rein Hypothetisch. Wobei ich eine 
\subsubsection{Rechenleistung teilen}
Obwohl schon ausgef\"uhrt wurde, dass es bei manchen Funktionen sinnvoller ist wie bei anderen, so soll es gerade bei rechenintensiven Anforderungen m\"oglich sein Rechenleistung auszulagern und zwischen Modulen zu teilen. Soll der Roboter Beispielsweise eine Treppe besteigen und scannt vorher Treppen und Abstand und berechnet die Fu\ss{}bewegung im Stand vor der Ausf\"uhrung, so kann es sinnvoll sein dies zu einem gro\ss{}teil auf dem Leistungsf\"ahigerem Server auszulagern. Au\ss{}erdem kann so ein Leistungssch\"acherer Prozessor in Module verbaut werden um Strom, oder eben eventuell Akku zu sparen. Trotzdem soll es m\"oglich sein zwischen allen Modulen Rechenleistung zu teilen, wodurch mehrere Roboter beispielsweise ebenfalls wie ein Clustercomputer agieren, wenn beispiielsweise ein Roboter gerade am Laden ist und womit nicht rechnen muss, so kann dieser einem anderen Roboter aushelfen, der Ressourcen ben\"otigt. 
\subsubsection{Server}
Der Server soll Ansprechpartner f\"ur jedes Modul sein, Wissen und Informationen sammeln und f\"ur alle Module verf\"ugbar machen, updates verteilen und in der Lage sein Module zu \"uberwachen und zu steuern. Ob und wie weit eine Anbindung an das Internet oder generell Anschluss an externe Daten (Kalender, Uhrzeit, etc. ) stattfindet wird noch diskutiert, M\"oglichkeiten, die sich dadurch er\"offnen w\"aren zwar grenzenlos, Ger\"ate mit derart vielen M\"oglichkeiten an das Internet anzubinden ist aus Sicherheitsaspekten jedoch zumindest diskusionsw\"urdig. Eventuell wird es es eine nachinstallierbare Internetextension geben, so dass eine solche Nutzung optional ist. 
\subsubsection{Clean Code}
Prinzipiell werden trotzdem anf\"anglichem Mehraufwand nur Releases mit sauberem Code, sinnvollen Kommentaren und guter Dokumentation herausgegeben. Ein Code der von externen Personen nicht lesbar oder verst\"andlich ist, kann nicht verbessert werden. Dar\"uber hinaus, einmal ein Hack, kann oft nur mit noch einem Hack benutz werden, und dieser mit noch einem Hack, und schlussendlich hat man ab einem gewissen Grad an Komplexit\"at ein unbrauchbares System, daher unser Motto, keine Kompromisse (Au\ss{}er zu pers\"ohnlichen testzweckend nat\"urlich in Branches)
\subsubsection{Sicherheit}
F\"ur Netzwerke und zu Speicherung der Daten werden nur State-of-the-Art Open Source Verschl\"usslungen und Sicherheitsverfahren verwendet.

\subsection{Hardware}
\subsubsection{Modularit\"at}
\"Ahnlich dem Linux Prinzip "Everything's a File" Soll im System gelten "Everything's a Module", womit das System erweitert werden kann und Teile ausgetauscht werden k\"onnen ohne dass Probleme entstehen. Ausgehend davon, dass der Roboter selbst aus verschiedenen Modulen besteht (Kopf, Beine, Arme), sollten somit auch Problemlos Module ausgetauscht werden k\"onnen, beispielsweise ein Arm mit besseren H\"anden, Beine durch ein Fahrgestell etc. Dar\"uber hinaus sollten diese Module Problemlos miteinander Kommunizieren k\"onnen, als w\"are alles ein System. Zum Beispiel, im System ist Heimautomation integriert, auch hier gelten alle Teile als Module, so sollte der Roboter die \"Uberwachunskameras zur Sicht und Orientierung nutzen k\"onnen. Dar\"uber hinaus sollten auch alle Module individuell benutzbar sein, wenn jemand Beispiels weise nur ein Arm baut und an diesem entwickeln will, so soll die m\"oglich sein, Will jemand nur einen Kopf f\"ur Sprach- und Gesichtserkennung bauen wollen so soll die m\"oglich sein. 
\subsubsection{G\"unstige Hardware}
Nur wenn jeder der will mitentwickeln kann, ist das Projekt sinnvoll. Somit muss nicht nur der Quellcode, sondern auch die Hardware verf\"urbar sein. Sind jedoch die Baupl\"ane im Internet und das Bauverfahren zu teuer, so ist dieses Prinzip nicht gegeben. Ziel muss es sein einen Kompromiss aus Leistung und Preis zu schaffen. Angestrebtes Ziel f\"ur den Preis in der Standardversion sind 150 bis 200 Euro. Man kann davon ausgehen, dass sp\"altere Versionen/Baupl\"ane durch mehr Erfahrung g\"unsicher werden und im Umkehrschluss erste Prototypen noch teurer sind. Komplett sets, zusammengebaut und in Einzelteilen werden gegen eine kleine Geb\"uhr m\"oglicherweise von uns bereit gestellt.  

\end{document}  